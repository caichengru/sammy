% Options for packages loaded elsewhere
\PassOptionsToPackage{unicode}{hyperref}
\PassOptionsToPackage{hyphens}{url}
%
\documentclass[
]{article}
\usepackage{amsmath,amssymb}
\usepackage{lmodern}
\usepackage{iftex}
\ifPDFTeX
  \usepackage[T1]{fontenc}
  \usepackage[utf8]{inputenc}
  \usepackage{textcomp} % provide euro and other symbols
\else % if luatex or xetex
  \usepackage{unicode-math}
  \defaultfontfeatures{Scale=MatchLowercase}
  \defaultfontfeatures[\rmfamily]{Ligatures=TeX,Scale=1}
\fi
% Use upquote if available, for straight quotes in verbatim environments
\IfFileExists{upquote.sty}{\usepackage{upquote}}{}
\IfFileExists{microtype.sty}{% use microtype if available
  \usepackage[]{microtype}
  \UseMicrotypeSet[protrusion]{basicmath} % disable protrusion for tt fonts
}{}
\makeatletter
\@ifundefined{KOMAClassName}{% if non-KOMA class
  \IfFileExists{parskip.sty}{%
    \usepackage{parskip}
  }{% else
    \setlength{\parindent}{0pt}
    \setlength{\parskip}{6pt plus 2pt minus 1pt}}
}{% if KOMA class
  \KOMAoptions{parskip=half}}
\makeatother
\usepackage{xcolor}
\IfFileExists{xurl.sty}{\usepackage{xurl}}{} % add URL line breaks if available
\IfFileExists{bookmark.sty}{\usepackage{bookmark}}{\usepackage{hyperref}}
\hypersetup{
  hidelinks,
  pdfcreator={LaTeX via pandoc}}
\urlstyle{same} % disable monospaced font for URLs
\usepackage[margin=1in]{geometry}
\usepackage{graphicx}
\makeatletter
\def\maxwidth{\ifdim\Gin@nat@width>\linewidth\linewidth\else\Gin@nat@width\fi}
\def\maxheight{\ifdim\Gin@nat@height>\textheight\textheight\else\Gin@nat@height\fi}
\makeatother
% Scale images if necessary, so that they will not overflow the page
% margins by default, and it is still possible to overwrite the defaults
% using explicit options in \includegraphics[width, height, ...]{}
\setkeys{Gin}{width=\maxwidth,height=\maxheight,keepaspectratio}
% Set default figure placement to htbp
\makeatletter
\def\fps@figure{htbp}
\makeatother
\setlength{\emergencystretch}{3em} % prevent overfull lines
\providecommand{\tightlist}{%
  \setlength{\itemsep}{0pt}\setlength{\parskip}{0pt}}
\setcounter{secnumdepth}{-\maxdimen} % remove section numbering
\ifLuaTeX
  \usepackage{selnolig}  % disable illegal ligatures
\fi

\author{}
\date{\vspace{-2.5em}}

\begin{document}

\begin{verbatim}
##  [1] "1. 根據師長在課堂上提供的參考碼,在「<-」的左邊是否可以用中文名稱當作是物件(某一個R環境裡的東西)的名字(請答「是」或是「否」,不用寫括號)?\n\n\n"                                                                                        
##  [2] "2. 根據師長在課堂上提供的參考碼,師長建議一個R字可以協助把「.html」檔案變成「.pdf」檔案,當使用這一個字的時候,我們會在括號內提供幾個檔案名稱(請用阿拉伯數字作答)?\n\n\n"                                                               
##  [3] "3. 根據師長在課堂上提供的參考碼,哪一個R字可以把「Numextract(\"D1234567\")」得到的「\"1234567\"」變成整數?\n\n\n"                                                                                                                       
##  [4] "4. 根據師長在課堂上提供的參考碼,哪一個R字可以把「某一個字」讀到的「.xlsx」或是「.xls」檔案後的「物件(某一個R環境裡的東西)」變成「data.frame」?\n\n\n"                                                                                  
##  [5] "5. 根據師長在課堂上提供的參考碼,當程式碼的「某一個字」讀到「.xlsx」或是「.xls」檔案之後,師長要求把「這一個物件(某一個R環境裡的東西)」變成「data.frame」,在那一個R字的括號裡面,師長把哪一個字設定為「FALSE」?\n\n\n"                 
##  [6] "6. 根據師長在課堂上提供的參考碼,師長建議一個協助讀取「.xlsx」或是「.xls」檔案的套件,當程式需要此套件進入「RStudio」的時候(假設程式已經呼叫過一次了),請問要用哪一個字?\n\n\n"                                                         
##  [7] "7. 根據師長在課堂上提供的參考碼,師長建議哪一個R字可以協助把「.html」檔案變成「.pdf」檔案?\n\n\n"                                                                                                                                       
##  [8] "8. 根據師長在課堂上提供的參考碼,哪一個R字可以協助取得某一個欄位的次數分配表?\n\n\n"                                                                                                                                                    
##  [9] "9. 根據師長在課堂上提供的參考碼,哪一個R字可以協助「『重複執行』『產生檔案夾』這一個動作」?\n\n\n"                                                                                                                                      
## [10] "10. 根據師長在課堂上提供的參考碼,哪一個R字可以協助讀取「.xlsx」或是「.xls」檔案?\n\n\n"                                                                                                                                                
## [11] "11. 根據師長在課堂上提供的參考碼,類似這樣的一句話,「Q1[,]」,欄位名稱要放在「,」的「左邊」還是「右邊」(請用中文作答,不用寫括號)?\n\n\n"                                                                                              
## [12] "12. 根據課堂上的要求,師長希望每一位同學開啟「學習本課程」的專案叫一樣的名字,然後把「師長開放出來的參考碼」跟「自行開發的參考碼」,甚至是「網路網友提供的參考碼」,都放在哪一個檔案夾內(請用英文作答)?\n\n\n"                          
## [13] "13. 根據課堂上的要求,師長希望每一位同學開啟「學習本課程」的專案叫一樣的名字,然後把「各種管道找到、取得的參考文獻」,都放在哪一個檔案夾內(請用英文作答)?\n\n\n"                                                                        
## [14] "14. 根據課堂上的要求,師長希望每一位同學開啟的專案,在哪一個檔案夾下方(請用中文作答)?\n\n\n"                                                                                                                                            
## [15] "15. 根據師長在課堂上提供的參考碼,哪「一串符號」可以把隨機取得某一張表全部欄位名稱其中三個的答案,放入「vars」?\n\n\n"                                                                                                                  
## [16] "16. 根據師長在課堂上提供的參考碼,哪一個R字可以把某一個整數設定為隨機種子?\n\n\n"                                                                                                                                                       
## [17] "17. 根據師長在課堂上提供的參考碼,哪一個R字可以協助讀取「.rds」檔案?\n\n\n"                                                                                                                                                             
## [18] "18. 根據課堂上的要求,師長希望每一位同學開啟「學習本課程」的專案叫一樣的名字,然後把「師長開放出來的數據集」跟「自行入校園收集到的數據集」,都放在哪一個檔案夾內(請用英文作答)?\n\n\n"                                                  
## [19] "19. 根據師長在課堂上提供的參考碼,哪一個R字可以取得某一張表全部的「欄位名稱」?\n\n\n"                                                                                                                                                   
## [20] "20. 根據課堂上的要求,師長希望每一位同學開啟的專案,叫什麼名字?\n\n\n"                                                                                                                                                                  
## [21] "21. 根據師長在課堂上提供的參考碼,師長建議哪一個套件協助讀取「.xlsx」或是「.xls」檔案?\n\n\n"                                                                                                                                           
## [22] "22. 根據師長在課堂上提供的參考碼,哪一個R字可以產生「新的檔案夾」?\n\n\n"                                                                                                                                                               
## [23] "23. 根據師長在課堂上提供的參考碼,哪一個R字可以「隨機」取得某一張表全部欄位名稱的其中三個?\n\n\n"                                                                                                                                       
## [24] "24. 根據師長在課堂上提供的參考碼,師長建議一個協助讀取「.xlsx」或是「.xls」檔案的套件,請問第一次呼叫此套件進入「RStudio」要用哪一個字?\n\n\n"                                                                                          
## [25] "25. 根據師長在課堂上提供的參考碼,師長建議哪一個套件協助把「.html」檔案變成「.pdf」檔案?\n\n\n"                                                                                                                                         
## [26] "26. 根據師長在課堂上提供的參考碼,在「.Rmd」檔案裡,如果要顯示產生答案的程式碼,哪一個英文字要設定為「TRUE」?\n\n\n"                                                                                                                    
## [27] "27. 根據師長在課堂上提供的參考碼,如果執行過「vars <- c(\"使用英文教科書,修習財管困難的地方\", \"學習財金的基本知識,選修財管的目的為何\", \"出生月\")」,那這句話「vars[3]」會抓到「vars」的哪一個(請用中文作答,不用寫雙引號)?\n\n\n"
## [28] "28. 根據師長在課堂上提供的參考碼,哪一個R字可以協助把「某一張表」存成「.rds」檔案?\n\n\n"                                                                                                                                               
## [29] "29. 根據課堂上的要求,師長希望每一位同學開啟「學習本課程」的專案叫一樣的名字,然後準備一個「垃圾夾、回收桶檔案夾」,請問叫什麼名字(請用英文作答)?\n\n\n"                                                                                
## [30] "30. 根據課堂上的要求,師長希望每一位同學開啟「學習本課程」的專案叫一樣的名字,然後把算出來的「次數分配表」,都放在「output」之下的哪一個檔案夾內(請用英文作答)?\n\n\n"
\end{verbatim}

\end{document}
